% ------------------------------------------------------------------------
% Name: Introduction
% Author: Emmanuel SCHWARTZ
% Date: November 2016
% Description: Introduction Chapter for my master thesis paper
% ------------------------------------------------------------------------

\section{Motivation}

Reasons why we decided to work on this topic
\paragraph{}
It is often considered that the history of electronic mail (or e-mail) begins in 1965, at a time when the Internet did not even exist yet. By that time, the first exchanges of messages was only possible between users on private networks were set up. One of the first systems to allow message exchange was the Competent Time-Sharing System (CTSS) of the famous Massachusetts Institute of Technology (MIT), although this paternity has also been claimed by System Development Corporation SDC) and its own Time-Sharing System created for the Q32, a computer specially manufactured by IBM for the US Air Force.
\paragraph{}
However, e-mail is only really born from the creation of the ARPAnet network, the ancestor of the Internet. In 1971, after writing some 200 lines of code in order to create two applications, SNDMSG nas READMAIL, the engineer named Raymond Samuel Tomlinson could sent the first email of history to himself. Some times later, 
Tomlinson found a way for the program to easily differentiate a local message from a network message: the symbol @ was born. It was a simple way to ​​dissociate a user name and host name with the only character that was not used in any proper name nor, above all, in any company name. The first "netmail" test was sent with only content "QWERTYUIOP", the first line of character of the English keyboard.
\paragraph{}
The email was so successful that it quickly became unthinkable for users of the ARPAnet network to do without it. As a result, the software quickly became the "killer app"  of the ARPAnet network, and developers focused either on improving the program and its transfer protocol, or creating their own solutions. In 1992, a great improvment was made: the world's first-ever email attachment, sent by the researcher Nathaniel Borenstein, where we could see a adorable photo of his barbershop quartet, The Telephone Chords. This was made possible thanks to MIME (Multipurpose Internet Mail Extensions),a internet standard that extends the data format of e-mails.
\paragraph{}
Fourty years later, despite the creation of Instant Messaging, or some years later, social networks, e-mails are still very popular: 183 billion of them are sent every day! If e-mails spam remains a major problem, e-mails has to face new challenges: Reliability \& Privacy. 
\paragraph{}
When a person is sending an e-mail, she expects that her message will be received successfully by the
intended recipient. For most cases, it does, but sometimes, for the following reasons, it does not:
\begin{easylist}[enumerate]
\ListProperties(Hide=100, Hang=false, Progressive=3ex, Style*=-- ,
Style2*=$\bullet$ ,Style3*=$\circ$ ,Style4*=\tiny$\blacksquare$ )

& The design of e-mail: Two users does not have to be online at the same time in order to communicate. This is called asynchronious communications. This is made possible by the mail servers, accepting messages from sources and attempt to relay them towards, or deliver them to, the recipient. In order to do so, e-mails have to jump from an server to an other: Some e-mails might get lost during these operations, for various reasons (busy servers, e-mails deferral, rejected e-mails)

& The exponential groth of e-mail spam has forced the use of e-mail rejection, intended to identify and separate legitimate e-mails from junk e-mails. Unfortunately, this has turned out into a false positive problem: Some legimate e-mails are considered as junk, this means the user thinks he did not receive the e-mail.
Some solutions try to solve these problems such as RE: Reliable Email[*], which tries to create an intelligent filter for e-mails, or tools that responds automatically to undeliverability by persisting with retransmission or retransmitting to alternate recipients [*]
\end{easylist}
\paragraph{}
Alongside with reliability, e-mail is facing one of the biggest challenges: Privacy. 
 
\section {Application Scenario}

E-Mail case scenario.
Thunderbird + EnigMail

\section{Objective}

What do we want to achieve?

\section{Overview}

Presentation of the upcoming chapters.