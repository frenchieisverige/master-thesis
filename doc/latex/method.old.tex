% -----------------------------------------------------------
%
%
% Method
%
%
% -----------------------------------------------------------

% -----------------------------------------------------------
%
% -----------------------------------------------------------
\section{Einbinden einer Grafik}

\subsection{Standard-Grafik}

Ein Leuchtturm in den typischen Farben ist in Abbildung \ref{meth-img-leuchtturm}
gezeigt.

\begin{figure}[htb]
  \centering
  \includegraphics[height=5cm]{leuchtturm.jpg}
  \caption{\label{meth-img-leuchtturm}Der Leuchtturm von irgendwo.}
\end{figure}

Zwei weitere Leuchtt�rme sind in Abbildung \ref{meth-img-weitere} gezeigt.

\begin{figure}[htb]
\begin{minipage}[b]{.45\linewidth} 
  \centering
  \includegraphics[height=4cm]{leuchtturm.jpg}
\end{minipage}
\hspace{.05\linewidth}
\begin{minipage}[b]{.45\linewidth} 
  \centering
  \includegraphics[height=4cm]{leuchtturm.jpg}
\end{minipage}
\caption{\label{meth-img-weitere}Weitere Leuchtt�rme.}
\end{figure}

% -----------------------------------------------------------
%
% -----------------------------------------------------------
\subsection{Mathematik}

\begin{lemma}[covariance enclosure]
The covariance ${\rm cov(X_a)}$ is smaller in all directions than ${\rm cov}(X_b)$
if and only if ${\rm cov}(X_b) \succeq {\rm cov}(X_a)$.
\end{lemma}

\begin{proof}
So-called $k$-sigma contours provide a convenient graphical representation of a random variable $X_a$.
The $k$-sigma contour of $X_a$ is defined by the points

\begin{displaymath}
({\bf x}-E(X_a))^T {\rm cov}(X_a)^{-1} ({\bf x}-E(X_a)) = k
\end{displaymath}

This term defines an ellipse for two dimensions respectively an hyperellipsoid for higher dimensions.
The covariance ${\rm cov}(X_a)$ is smaller than ${\rm cov}(X_b)$ in all directions if the corresponding
$k$-sigma contour of $X_a$ is completely enclosed by the $k$-sigma contour of $X_b$. This is equivalent to

\begin{eqnarray}
({\bf x}\!-\!{\bf c})^T {\rm cov}(X_a)^{-1} ({\bf x}\!-\!{\bf c}) \! & \! \geq \! & \! ({\bf x}\!-\!{\bf c})^T {\rm cov}(X_b)^{-1} ({\bf x}\!-\!{\bf c}) \nonumber \\
{\rm cov}(X_a)^{-1} & \! \succeq \! & {\rm cov}(X_b)^{-1} \nonumber \\
{\rm cov}(X_b) & \! \succeq \! & {\rm cov}(X_a) \nonumber
\end{eqnarray}

\end{proof}

% -----------------------------------------------------------
%
% -----------------------------------------------------------
\subsection{Beispiel URL}

\urldef{\homepage-schlegel}{\url}{http://www.rz.fh-ulm.de/~cschlege}

Hier ist der Link: \homepage-schlegel

% -----------------------------------------------------------
%
% -----------------------------------------------------------
\subsection{Tabelle}

\begin{table}[htb]
\centering
\begin{tabular}{llll}\hline
No. & Supported Feature   & Octet & Bit \\ \hline
0   & Flow Control Mode   & 0     & 0   \\
1   & Retransmission Mode & 0     & 1   \\
2   & Bi-directional QoS  & 0     & 2   \\
31  & Reserved for feature mask ext. &3 &7 \\ \hline
\end{tabular}
\caption{\label{meth-tab-1}Eine Tabelle.}
\end{table}

% -----------------------------------------------------------
%
% -----------------------------------------------------------
\subsection{Literatur}

Eine ausf�hrliche Darstellung findet sich in \cite{schlegel-ars06}.

